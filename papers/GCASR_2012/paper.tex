\documentclass[conference]{IEEEtran}
% *** GRAPHICS RELATED PACKAGES ***
%
\ifCLASSINFOpdf
  \usepackage[pdftex]{graphicx}
  % declare the path(s) where your graphic files are
  % \graphicspath{{../pdf/}{../jpeg/}}
  % and their extensions so you won't have to specify these with
  % every instance of \includegraphics
  \DeclareGraphicsExtensions{.pdf,.jpeg,.png}
\else
  % or other class option (dvipsone, dvipdf, if not using dvips). graphicx
  % will default to the driver specified in the system graphics.cfg if no
  % driver is specified.
  \usepackage[dvips]{graphicx}
  % declare the path(s) where your graphic files are
  % \graphicspath{{../eps/}}
  % and their extensions so you won't have to specify these with
  % every instance of \includegraphics
  % \DeclareGraphicsExtensions{.eps}
\fi

\usepackage{pgf}
%\usepackage{tikz}
%\usetikzlibrary{arrows,automata}

\definecolor{darkgreen}{rgb}{0,0.7,0}

\newif\ifdraft
\drafttrue
%\draftfalse
\ifdraft
 \newcommand{\katznote}[1]{ {\textcolor{blue} { ***Dan:   #1 }}}
 \newcommand{\ketanote}[1]{{\textcolor{orange}  { ***Ketan:   #1 }}}
 \newcommand{\kriedernote}[1]{ {\textcolor{darkgreen}  { ***Scott:   #1 }}}
 \newcommand{\note}[1]{ {\textcolor{red}    {\bf #1 }}}
\else
 \newcommand{\katznote}[1]{}
 \newcommand{\kriedernote}[1]{}
 \newcommand{\note}[1]{}
\fi
% correct bad hyphenation here
%\hyphenation{op-tical net-works semi-conduc-tor}

\hyphenation{Queuing}

\begin{document}
%
% can use linebreaks \\ within to get better formatting as desired
\title{An Overview of Current and Future Computing Accelerator Architectures}

%\author{\IEEEauthorblockN{Auth1\IEEEauthorrefmark{1},
%Auth2\IEEEauthorrefmark{1}\IEEEauthorrefmark{1}, 
%Auth3\IEEEauthorrefmark{1},
%\IEEEauthorblockA{\IEEEauthorrefmark{1}Argonne National Laboratory}
%}}

\author{Scott J. Krieder\IEEEauthorrefmark{1},
Ioan Raicu\IEEEauthorrefmark{1}\IEEEauthorrefmark{2}\\
\IEEEauthorblockA{
\IEEEauthorrefmark{1}Department of Computer Science, Illinois Institute of Technology}
\IEEEauthorrefmark{2}MCS Division, Argonne National Laboratory
}


\maketitle

\begin{abstract}
Accelerator technologies are now quite common in Supercomputers, Clusters, Grids, and personal desktops. This work aims to provide an overview of the current technologies that are available today, and examine future accelerator technologies. This work examines the 3 major competitors in the Accelerator market including: NVIDIA, Intel, and AMD.
\end{abstract}

% no keywords
\begin{IEEEkeywords}
Accelerators, Coprocessors, GPGPGU, NVIDIA, AMD, Intel Xeon Phi.
\end{IEEEkeywords}

\IEEEpeerreviewmaketitle

\section{Introduction}
Hardware accelerators allow the machine to offload work from the host CPU to the Accelerator. The Accelerator then completes the computation and returns the solution back to the host CPU which relieves the host of previous compute cycles. In this work we analyze three major competitors in the HPC Hardware Accelerator market including: NVIDIA, AMD, and Intel. The motivation for examining accelerators revolves around Many-Task Computing(MTC).\cite{raicu2008toward} By enabling MTC on Accelerators we believe that can bridge the gap between clusters and GPU compute. MTC is an excellent candidate for the GPUs many power efficient cores. Some disadvantages that provide added motivation include: 1)Slow data transfers, 2)Lag induces communication, and 3)Programmability which requires additional languages. Our future work aims to address the shortcomings to provide efficient support for MTC on Hardware Accelerators.

\section{Accelerators}

\subsection{NVIDIA GPGPUs}
NVIDIA recently launched their newest GPU architecture called Kepler. \cite{NVIDIA_Kepler} Kepler provides many added benefits, but the 3 most noteworthy include: 1)SMX,  2)Dynamic Parallelism, and 3)Hyper Q.

Under the new architecture SMX compute elements can contain up to 192 cores per Streaming Multiprocessor compared to 32 cores per SM in previous architectures. This accounts for a 3X performance/watt.

Dynamic Parallelism allows a thread on the GPU to spawn more threads. Under previous architectures a GPU thread had to be initiated from the host CPU resulting in increased communication back and forth across the PCI bus. Dynamic Parallelism allows for an extensive reduction in the number of communications across the PCI bus and therefore increased performance.

Finally, Hyper Q allows additional CPU cores to interact with the device. This eliminates host CPU idle time and adds to performance.

\subsection{AMD GPGPUs}
AMD GPUs provide an increase level of openness in regards to programmability. \cite{AMD_web} OpenCL is an open standard and there are any array of related works to support Java, Fortran, C++, .NET and Python. However, due to the abstract nature of OpenCL it might not provide the highest level of hardware performance and adds challenges to code reusability and maintenance.

\subsection{Intel Xeon Phi Coprocessor}
Text.\cite{Xeon_Phi_web}

\section{Future Work}
GPU Virtualization is one future work that we believe can provide increased GPU utilization and performance. For applications that cannot utilize an entire GPU, a portion of that GPU remains idle. A virtualized GPU would provide a small cluster for compute and enable separate applications and users to share hardware resources.

Workflow Systems have proven a highly successful solution for homogenous CPU based systems. Todays architectures are growing increasingly hybrid and by enabling Workflow Systems to support hardware accelerators such as NVIDIA GPUs we believe that we can increase application performance and system utilization.

\section{Conclusions}
In conclusion this work evaluates current generation hardware accelerators including the NVIDIA GPUs, AMD GPUs, and the Intel Xeon Phi. Running CUDA on NVIDIA GPUs is one of the most mature GPGPU solutions and provides high raw computational performance, however this does require code ported to the CUDA platform. The Intel Xeon Phi suffers from a lack of availability, but once this device is highly available it should bring large improvements in regards to accelerator programmability due to the familiar x86 environment. Finally, AMD GPUs provide a high level of openness in regards to programmability. AMD supports open standards such as OpenCL but may see difficulty in adoption within the HPC markets due to performance. The future of HPC will definitely involved Acclerators, but only time will tell which vendors win the war.

\bibliographystyle{IEEEtran}
\bibliography{ref}
\end{document}
